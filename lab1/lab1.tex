\problemset{Теория вероятностей и математическая статистика}
\problemset{Индивидуальное домашнее задание №1}	% поменяйте номер ИДЗ

\renewcommand*{\proofname}{Решение}

%%%%%%%%%%%%%% ЗАДАНИЕ №1 %%%%%%%%%%%%%%
%% Условие задания №1
\begin{problem}
        5 красных карточек и 7 синих карточек наудачу разложены по 4-м папкам. Найти вероятность того, что в каждой папке будут карточки двух цветов.
\end{problem}

%% Решение задания №1
\begin{proof}
    Пусть событие $A$ --- "В каждой папке карточки двух цветов". Разложим 5 красных карточек по 4-м папкам. ${{5 + 1 + (4 - 1) - 1} \choose {4 - 1}} = {8 \choose 3}$. Аналогично для синих карточек: ${{7 + 1 + (4 - 1) - 1} \choose {4 - 1}} = {10 \choose 3}$, следовательно общее число исходов будет равно: $\# \Omega = {8 \choose 3} \cdot {10 \choose 3} = 6720.$ Число благоприятных исходов будет считаться аналогично, только теперь нам необходимо, чтобы хотя бы 1 карточка была в папке. Для красных карточек: ${{5 - 1} \choose {4 - 1}} = {4 \choose 3}$, для синих: ${{7 - 1} \choose {4 - 1}} = {6 \choose 3}$, следовательно общее число благоприятных исходов будет равно: $\#A = {4 \choose 3} \cdot {6 \choose 3} = 80.$ Тогда по определению вероятности:\newline
    $\mathbb{P}A = \frac{\#A}{\#\Omega} = \frac{80}{6720} \approx 0,0119\newline$
    \newline
Ответ: 0,0119
\end{proof}

%%%%%%%%%%%%%% ЗАДАНИЕ №2 %%%%%%%%%%%%%%
%% Условие задания №2
\begin{problem}
	Прямые разбивают плоскость на полосы ширины 6. Определить вероятность того, что отрезок длины 2, наугад брошенный на плоскость, не пересечет ни одной прямой
\end{problem}

%% Решение задания №2
\begin{proof}
$\newline a = 3\newline r = 1\newline \mathbb{P}A$ --- вероятность, что отрезок не пересечет прямую. \newline $\mathbb{P}B$ --- вероятность, что отрезок пересечет прямую. \newline
\begin{figure}[H]
    \centering
    \includegraphics[width=0.5\linewidth]{1idz_1.png}
    \caption{Геометрическое представление задачи}
\end{figure}
	Возможные положения отрезка на плоскости полностью определяются положением центра отрезка и углом поворота отрезка относительно какого-либо направления – для удобства выберем в качестве направления прямую, параллельную исходным двум, и зададим положение отрезка углом между иголкой и этой прямой (он будет от 0 до $\pi$ обозначим его через $\phi$). Причём две эти переменные (положение центра и угол поворота) меняются независимо друг от друга. Обозначим через  расстояние от середины отрезка до ближайшей прямой, это расстояние находится в промежутке от 0 до $a$. Следовательно, можно посчитать меру всего пространства исходов. Это будет некий прямоугольник, одна сторона которого $\pi$, а другая $a$. Площадь этого прямоугольника:\newline
 $S = \pi \cdot a\newline$
 Посчитаем благоприятные исходы. Отрезок пересекает ближайшую прямую, если расстояние от середины отрезка до ближайшей прямой не превосходит $r \cdot \sin{\phi}\newline$
 Найдем площаль синусоиды $P$: \newline
 $P = \int_{0}^{\pi} r \cdot \sin{\phi} d \phi = 2 \cdot r\newline$
 \begin{figure}[H]
     \centering
     \includegraphics[width=0.5\linewidth]{1швя_2.png}
     \caption{График}
     \label{fig:enter-label}
 \end{figure}
 Для нахождения вероятность того, что отрезок пересечёт какую-нибудь прямую, необходимо разделить площадь синусоиды $P$ на площадь прямоуольника $S$:\newline
 $\mathbb{P}B = \frac{\#P}{\#S} = \frac{2 \cdot r}{a \cdot \pi} = \frac{2}{3 \cdot \pi}\newline$
 Вероятность того, что отрезок НЕ пересечет $\mathbb{P}A = 1 - \mathbb{P}B = 1 - \frac{2}{3 \cdot \pi} \approx 0,7878\newline$
 \newline
 Ответ: 0,7878
\end{proof}

%%%%%%%%%%%%%% ЗАДАНИЕ №3 %%%%%%%%%%%%%%
%% Условие задания №3
\begin{problem}
	В первой урне находится 8 белых и 8 черных шаров, во второй --- 8 белых и 18 черных шаров. Одновременно из первой и второй урн вытаскивают по шару, перемешивают и возвращают по одному в каждую урну. Затем из каждой урны вытаскивают по шару. Они оказались одного цвета. Определить вероятность того, что в первой урне осталось столько же белых шаров, сколько было вначале.
\end{problem}

%% Решение задания №3
\begin{proof}
	Определим полную группу событий:$\newline$
$H1$ --- "Из первого ящика достали белый и вернули белый"$\newline$
$H2$ --- "Из первого ящика достали белый и вернули черный"$\newline$
$H3$ --- "Из первого ящика достали черный и вернули белый"$\newline$
$H4$ --- "Из первого ящика достали черный и вернули черный"$\newline$
Событие $A$ --- "В первом ящике осталось столько же белых шаров".$\newline$
$\mathbb{P}(H1) = \frac{8}{16} \cdot \frac{8}{26} = \frac{2}{13} \newline$
$\mathbb{P}(H2) = \frac{8}{16} \cdot \frac{18}{26} = \frac{9}{26} \newline$
$\mathbb{P}(H3) = \frac{8}{16} \cdot \frac{8}{26} = \frac{2}{13} \newline$
$\mathbb{P}(H4) = \frac{8}{16} \cdot \frac{18}{26} = \frac{9}{26} \newline$

$\mathbb{P}(A|H1) = \frac{8}{16} \cdot \frac{18}{26} = \frac{9}{26}$ (чч) \newline
$\mathbb{P}(A|H2) = 0$ (т.к. в 1-ой урне недостаточно белых) \newline
$\mathbb{P}(A|H3) = \frac{9}{16} \cdot \frac{7}{26} = \frac{63}{416}$ (бб) \newline
$\mathbb{P}(A|H4) = \frac{8}{16} \cdot \frac{18}{26} = \frac{9}{26}$ (чч) \newline

\begin{center}
		\begin{tabular}{|c|c|c|c|c|c|}
			\hline
			$ Hi $  & $ H1 $ & $H2$ & $ H3 $ & $ H4 $ & $ \Sigma $ \\ \hline
			$ \mathbb{P}(Hi) $ & $\frac{2}{13}$ & $ \frac{9}{26}$ & $\frac{2}{13}$&$\frac{9}{26}$  & 1        \\ \hline
                $ \mathbb{P}(A|Hi) $ &$\frac{9}{26}$& 0&$\frac{63}{416}$&$\frac{9}{26}$& ---
                \\ \hline
		\end{tabular}
	\end{center}

 Найдем вероятность события $A$, используя формулу полной вероятности:\newline
 $\mathbb{P}(A) = \sum_{i=0}^4  \mathbb{P}(A|Hi) \cdot \mathbb{P}(Hi) = \frac{2}{13} \cdot \frac{9}{26} + 0 + \frac{2}{13} \cdot \frac{63}{416} + \frac{9}{26} \cdot \frac{9}{26} = \frac{531}{2704} \approx 0,1964\newline$

 Ответ: 0,1964

\end{proof}

%%%%%%%%%%%%%% ЗАДАНИЕ №4 %%%%%%%%%%%%%%
%% Условие задания №4
\begin{problem}
	При посылке сообщения, состоящего из 20-ти символов вероятность искажения каждого символа равна 0.01. Для надежности сообщение передается трижды. При этом известно, что при первой передаче были точно искжены первые 11 символов, а при последней передаче были искажены символы с 10-го по 20-й. Определить вероятность того, что на основании трех передач сообщение удастся восстановить.  
\end{problem}

%% Решение задания №4
\begin{proof}
	Обозначим события, вероятность которого требуется найти: 
 $A$ --– на основании трёх передач сообщение удастся восстановить.\newline
 $B$ --- есть неискаженный символ среди двух сообщений\newline
 $C$ --- получены символы 1--9\newline
 $D$ --- получены символы 10--11\newline
 $E$ --- получены символы 12--20\newline
 $p = 0,01$ --- вероятность искажения любого символа.\newline
 В событиях $C$ и $E$ в одном из сообщений символы искажены, поэтому там считаем вероятность из двух сообщений. В событии $D$ символы не искажены лишь в одном сообщении, поэтому там будем считать вероятность из одного сообщения:\newline
 $\mathbb{P}(B) = 1 - 0,01^2 = 0,9999\newline$
 $\mathbb{P}(C) =\mathbb{P}(B)^9 = 0,9991\newline$
 $\mathbb{P}(D) = (1 - 0,01)^2 = 0,9801\newline$
 $\mathbb{P}(E) = \mathbb{P}(B)^9 = 0,9991\newline$
 $\mathbb{P}(A) = \mathbb{P}(C) \cdot \mathbb{P}(D) \cdot \mathbb{P}(E) = 0,9783 \newline$
 \newline
 Ответ: 0,9783
\end{proof}

%%%%%%%%%%%%%% ЗАДАНИЕ №5 %%%%%%%%%%%%%%
%% Условие задания №5
\begin{problem}
	Вероятность успеха в схеме Бернулли равна $\frac{1}{4}$. Проводится 500 испытаний. Написать точную формулу и вычислить приближенную вероятность того, что число успехов попадет в интервал 115--135. 
\end{problem}

%% Решение задания №5
\begin{proof}
$\newline p = \frac{1}{4}\newline$
$n = 500\newline$
$np = 125 > 10$, следовательно будем использовать алгоритм Муавра-Лапласа. Сначала найдем общую формулу:\newline
1) $\mathbb{P}(A) = \sum_{m=115}^{135} {500 \choose m} \cdot \left(\frac{1}{4}\right)^m \cdot \left(\frac{3}{4}\right)^{500-m}\newline$
Теперь найдем вероятность:\newline
2) $\Phi(x2) - \Phi(x1) = \Phi(\frac{135 - 125}{\sqrt{125 \cdot 0,75}}) - \Phi(\frac{115 - 125}{\sqrt{125 \cdot 0,75}}) = \Phi(1,0328) \cdot 2 = 0,697\newline$
\newline
Ответ: 0,697

\end{proof}